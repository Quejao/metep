\documentclass{article}
\usepackage[utf8]{inputenc}
\usepackage{graphicx}
\usepackage{cleveref}



\title{Análise do uso de gamificação voltada à educação}
\author{Leonardo Corrá}
\date{Maio 2017}

\begin{document}

\maketitle

%\begin{figure}
	
%	\center
%	\includegraphics[scale=.2]{images.png}
%	\caption{Logo da UTFPR.}
	
%\end{figure}


\section{Introdução}
\large Na área da educação, um dos grandes problemas, em qualquer tipo de ensino, é desenvolver o interesse do aluno para a área a ser ensinada, já que um aluno interessado trabalha com mais afinco. Dito isso, pesquisas atrás de métodos para atrair a atenção dos alunos tem crescido nos últimos tempos chegando assim na gamificação.

\section{Método}

\section{Resultados}

\subsection{Estudo 1}

\subsection{Estudo 2}

\section{Conclusão}

\cite{cardoso:2014}
\cite{seaborn:2014}
\cite{dixon:2011}
\cite{huang:2013}
\cite{isotani:2014}
\cite{oliveira:2016}
\cite{ibanez:2014}
\cite{morrison:2014}
\cite{hamari:2014}
\bibliographystyle{plain}
\bibliography{referencias}

\end{document}
